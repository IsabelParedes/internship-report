\chapter{Conclusion}

    As the brief twelve weeks come to an end, it is time to reflect on all the progress achieved during this time. 


    \section*{The Challenges}

        At the beginning of this internship, I believed that I had an idea of what software development involved, but I quickly discovered I had no idea of what was really involved. Thus, starting with essentially zero knowledge, I had to learn how to develop software on the fly. 

        The main recurring problem was definitely getting development environments set up properly. Even after having set up plenty of environments by now, occasionally I still find errors which I am unable to resolve and I resort to removing everything and starting over which, more often than not, solves the issue.

        Another challenge that I faced was simply not being aware of all the tools available which make coding and debugging a thousand times easier. Regardless of how many tutorials I read through, not every resource will cover every single tool. So I am left with randomly stumbling upon useful tools when trying to fix an error, or learning about these tools from mentors which is often the quickest path.

        Finally, as would be common in any other software internship I had to learn a new programming language. Although having a background in programming enables one to more quickly learn a new language, being confident in that language still takes time and a large amount of practice. At the beginning of the internship I knew the basics of JavaScript, but working on developing extensions for JupyterLab felt like jumping into the belly of the beast for TypeScript. But after the last few weeks of practice, I am finally gaining some confidence to at least be able to read through the code and know how it works.

    \section*{The Lessons}

        One of the most pressing lessons I learned during this time period was the importance of dedicating time to writing documentation for any software project. Having spent countless hours attempting to read through source code or figuring things out by trial and error, it is my belief that everyone would benefit from even a minimal set of working examples or descriptions for each project.

        I have also learned that oftentimes it is more efficient to ask questions directly to other developers than to try and resolve everything individually. There were plenty of times when I would encounter an issue which would take up a large amount of time, but the solution was very simple and I could have arrived at the solution much sooner if I had simply asked. This is a skill that I am still finessing, knowing when to continue trying and knowing when to ask for help.

        Lastly, I learned about the importance of making connections with other developers and scientists or enthusiasts, specially in such niche communities. People are always more willing to help with issues when there already exists a history of established rapport.


    \pagebreak

    \section*{The Final Thoughts}

        To conclude, the overall internship experience has been overwhelmingly positive. In comparison to previous internships, this period has been the most intellectually stimulating. Not only were the projects directly related to my personal interests, but they also dove into new territories which gave me the opportunity for immense personal growth. I hope to continue this trajectory with my master's thesis in the following months.
